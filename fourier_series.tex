\documentclass[11pt]{article}
% decent example of doing mathematics and proofs in LaTeX.
% An Incredible degree of information can be found at
% http://en.wikibooks.org/wiki/LaTeX/Mathematics

% Use wide margins, but not quite so wide as fullpage.sty
\marginparwidth 0.5in
\oddsidemargin 0.25in
\evensidemargin 0.25in
\marginparsep 0.25in
\topmargin 0.25in
\textwidth 6in \textheight 8 in
% That's about enough definitions

\usepackage{amsmath}
\usepackage{elmath}
%\usepackage{babel-polish}
\usepackage[utf8]{inputenc}
\usepackage{graphicx}

\begin{document}
\author{Henryk Nowakowski}
\title{Szeregi Fouriera}
\maketitle


%TODO sprawdzić 1. raz


\begin{enumerate}

\item % Problem 1:

Ogólna postać:


\begin{align*}
f(x) \approx / = \frac{a_{0}}{2}+\sum_{n=1}^{\infty} a_{n}\cos(nx) + b_{n}\sin(nx))\\ \\
a_{n}= \frac{1}{\pi}\int_{-\pi}^{\pi} f(x) \cos(nx) dx,\quad n=0,1,2,3,... \\
\\ b_{n}= \frac{1}{\pi}\int_{-\pi}^{\pi} f(x) \sin(nx) dx, \quad n=1,2,3,... \\
Uwaga! \quad a_{0} = \frac{1}{\pi} \int_{-\pi}^{\pi} f(x) dx
\end{align*}


\line(1,0){250}

 For a function $f(x)$ periodic on an interval $[-L,L]$ instead of  $[-\pi,\pi]$, a simple change of variables can be used to transform the interval of integration from $[-\pi,\pi]$ to $[-L,L]$.

\begin{align*}
\\  \\
a_{n}= \frac{1}{L}\int_{-L}^{L} f(x) \cos(\frac{n\pi}{L}) dx,\quad
n=0,1,2,3,... \\ \\ b_{n}= \frac{1}{L}\int_{-L}^{L} f(x) \sin(\frac{n\pi x}{L}) dx,\quad
n=0,1,2,3,... \\
f(x) \approx / = \frac{a_{0}}{2}+\sum_{n=1}^{\infty} a_{n}\cos(\frac{n \pi x}{L}) + b_{n}\sin(\frac{n \pi x}{L})
\end{align*}


\item % Problem 2
Rozwinięcie w szereg sinusów:

\begin{align*}
    f(x) \approx \sum_{n=1}^{\infty} b_{n} \sin(nx) \\ \\
    b_{n} = \frac{2}{\pi} \int_{0}^{\pi} f(x)\sin(nx) dx
\end{align*}

Letting the range go to $L$,

\begin{align*}
    b_{n} = \frac{2}{L} \int_{0}^{L} f(x) \sin (\frac{n \pi x}{L}) dx
\end{align*}


Rozwinięcie w szereg cosinusów:

\begin{align*}
    f(x) \approx \frac{a_{0}}{2} \sum_{n=1}^{\infty} a_{n} \cos(nx) \\ \\
    a_{n} = \frac{2}{\pi} \int_{0}^{\pi} f(x)\cos(nx) dx \\ \\
    a_{0} = \frac{1}{\pi} \int_{-\pi}^{\pi} f(x) dx
\end{align*}

Letting the range go to $L$,

\begin{align*}
    a_{0} = \frac{2}{L} \int_{0}^{L} f(x) dx \\\\
    a_{n} = \frac{2}{L} \int_{0}^{L} f(x) \cos(\frac{n \pi x}{L})dx
\end{align*}

Przydatne wzorki:

\begin{align*}
    \cos\alpha \cos\beta = \frac{\cos(\alpha + \beta)+\cos(\alpha - \beta)}{2} \\ \\
    \sin\alpha \cos\beta = \frac{\sin(\alpha + \beta)+\sin(\alpha - \beta)}{2} \\\\
    \sin\alpha \sin\beta = \frac{\cos(\alpha - \beta)+\cos(\alpha + \beta)}{2} \\\\\\
\end{align*}





Wzorki do rozwijania w szereg Fouriera zespolony:

\begin{align*}
     e^{ix}=\cos x + i\sin x => \\\\
     \cos x = \frac{e^{ix} + e^{-ix}}{2}\\\\
     \sin x = \frac{e^{ix} - e^{-ix}}{-i2}\\\\
\end{align*}

Zespolony szereg Fouriera:

\begin{align*}
     f(x) \approx \sum_{-\infty}^{\infty} c_{n}e^{inx} \\\\
     c_{n} = \frac{1}{2 \pi} \int_{-\pi}^{\pi} f(x) e^{-inx}dx
\end{align*}


For a function periodic in $[\frac{-L}{2},\frac{L}{2}]$, these become

\begin{align*}
    f(x) = \sum_{x= -\infty}^{\infty} A_{n}e^{(2 \pi n x)/L}
    A_{n} = \frac{1}{L}\int_{-L/2}^{L/2} f(x) e^{-i(2 \pi n x)/L}
\end{align*}


Żródła:\\
\url{http://mathworld.wolfram.com/FourierSeries.html}\\
\url{http://mathworld.wolfram.com/FourierSineSeries.html}\\
\url{http://mathworld.wolfram.com/FourierCosineSeries.html}




\end{enumerate}
\end{document}
